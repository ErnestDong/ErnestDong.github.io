% !TeX root = main.tex
\subsection{宏观因素}
\begin{frame}{政策与环境}
	货币政策、财政政策
	\begin{itemize}
		\item 疫情 \cite{mirza2020impact}
		\item 贸易战
		\item 经济下行 \citet{bali2021macroeconomic}
	\end{itemize}
\end{frame}


\begin{frame}{市场流动性}
	市场流动性偏紧,企业正常融资无法接续。如永煤违约之后相似主体如清控、紫光、冀中能源等发债遇到困难,最终部分主体走向违约。
\end{frame}

\subsection{中观因素}
\begin{frame}{行业景气和行业政策}
	房地产政策、光伏补贴政策都可能影响企业违约。

	顺周期行业中可能因为行业景气循环判断失误,经营出现问题,影响偿债能力。
\end{frame}
\begin{frame}{地域}
	区域风险传染,可能的路径有股权、人事、地区互保等情况,一荣俱荣一损俱损。
\end{frame}
\subsection{微观因素}
\subsubsection{公司层面}
\begin{frame}{公司治理}
	\begin{itemize}
		\item 高管变动\cite{林晚发2018高管任职经历的得与失}
		\item 母子关系
		\item 客户集中度\cite{王雄元2017客户集中度与公司债二级市场信用利差}
	\end{itemize}
\end{frame}

\begin{frame}{经营}
	\begin{itemize}
		\item 业绩巨亏
		\item 非标违约
		\item 对外担保
		\item 股权质押
		\item 客户集中度 \citep{王雄元2017客户集中度与公司债二级市场信用利差}
	\end{itemize}
\end{frame}

\begin{frame}{财务}
	\begin{itemize}
		\item 杠杆\cite{王永钦2019杠杆率如何影响资产价格}
		\item 营业收入
		\item 货币资金
	\end{itemize}
\end{frame}

\begin{frame}{发行人主观意愿}
	难以量化。

	\begin{quote}
		永城煤电在2020年10月20日发行“20永煤MTN006”、账面仍有大量货币资金时选择“20永煤SCP003”违约逃废债。

		花样年在账面留有大量现金时因行业景气结束选择躺平放弃履约。
	\end{quote}

\end{frame}
\subsubsection{债项层面}
\begin{frame}{债券分类}
	我国信用债针对发行面向对象可分为公募债和私募债。针对公募债和私募债,监管要求的信息披露、发行条件等也有所不同。

	因此相似期限的中期票据(公募债的一种)违约率通常比定向债务融资工具(私募债)高。
\end{frame}
\begin{frame}{收益率、折算率、评级}
	高收益率伴随而来的是高风险。正是因为承担了较大的风险,投资者才会要求债券给予较大的收益率补偿。
\end{frame}
% \begin{frame}{标准券折算率}
% 不透明,放弃
% \end{frame}
