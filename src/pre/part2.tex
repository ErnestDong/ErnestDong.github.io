% !TeX root = main.tex
\subsection{信用利差}
\begin{frame}{影响因素}
	\begin{itemize}
		\item 信用评级\cite{陈关亭2021多重信用评级与债券融资成本}\cite{黄小琳2017债券违约对涉事信用评级机构的影响——基于中国信用债市场违约事件的分析}\cite{寇宗来2015中国的信用评级真的影响发债成本吗}\cite{寇宗来2021私有信息}
		\item 公司治理\cite{常莹莹2019环境信息透明度与企业信用评级}
		\item 市场流动性\cite{钟宁桦2018散户投资者如何影响债券价格}
		      % \item 流动性溢价
		      % \item 跨市场流动性
		\item 宏观政策\cite{韩鹏飞2015政府隐性担保一定能降低债券的融资成本吗}\cite{汪莉2015政府隐性担保}
		      \begin{itemize}
			      \item 财政政策\cite{梅冬州2021财政扩张}、
			      \item 货币政策\cite{王博2019货币政策不确定性}
			            宏观环境\citet{bai2019common}
		      \end{itemize}
	\end{itemize}
	\ernest{}
\end{frame}

\subsection{风险传染}
\begin{frame}{传染路径}
	在美国,违约风险传染主要是宏观层面和微观层面的 \citet{azizpour2018exploring}
	\begin{itemize}
		\item 系统性风险\cite{苟文均2016债务杠杆与系统性风险传染机制——基于}
		\item 非系统性风险\cite{钟辉勇2016城投债的担保可信吗}
	\end{itemize}
	\ernest{}
\end{frame}
